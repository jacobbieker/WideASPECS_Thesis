\documentclass[twoside,single]{lion-msc}
\usepackage{multirow}
\usepackage{hyperref}
\usepackage{adjustbox}

\title{Wide ASPECS}
\author{Jacob Bieker}
\degree{Master of Science}   
\major{Astronomy}
\studentid{2153246}
\supervisor{Prof. Jacqueline Hodge}

\affiliation{Leiden Observatory, Leiden University} 
\address{P.O. Box 9513, 2300 RA Leiden, The Netherlands} 

\abstract{In this report, data from the Wide ASPECS survey is used to look at the clustering and halo masses of CO emitters within the GOODS-South region of the sky. The survey builds on the results of the ASPECS Pilot and Large Program surveys, cross-matching detected CO lines to galaxies. There is a clear increase in the two-point correlation function at higher fidelities for the CO line candidates.}

\begin{document}

\maketitle

\pagenumbering{roman}
\setcounter{page}{2}
\tableofcontents
\cleardoublepage

\pagenumbering{arabic}
\setcounter{page}{1}
\chapter{Introduction}

[FRIDAY: Get Prev. Work written for Survey and Clustering. End of Day: Remove two bands that are off for Appendix, Clean up Intro more, Add plots for clustering/nicer ones for Survey, Remove very low MAGPHYS results from survey plots, and plot distributions/how many are left. ]

[SATURDAY: Write better abstract, clean up intro more/Cite everything. Add more to Future Work based off past emails. Make last remaining Tables. Fill out Discussion on Survey, and Clustering based off current results. Fix COSMIC Volume in Matching. Run Matching for all lines, not just matched lines -> Remove checks, Make contours red and plot in 2 sigma intervals. Add all plots in Appendix.]. 

[SUNDAY: Incorporate Changes, Attempt running HALOFIT. Send off by 4pm to Jackie et al.] 

[TABLE OF r0 with Linear Interp, Gauss Interp, O to Max and 1.5 to 3.5]

Gas is the building block of stars and galaxies, and cold gas constitutes the majority of the interstellar medium (ISM). This gas can inform us about the evolution of galaxies and large scale structure in the universe, especially as observations cover larger and larger volumes of space. Using radio and submm telescopes, such as the Atacama Millimeter/Submillimeter Array (ALMA), this gas can be observed through looking for the transition lines of carbon monoxide (CO). In the Wide ALMA Spectroscopic Survey in the Hubble Ultra Deep Field (Wide ASPECS) survey, ALMA has been used to observe the largest volume of space yet for CO emissions, on the order of 550,000 cMpc, enabling for the first time the ability to look at these CO emitters in bulk. [NEEDS WORK]

This project is focused on using Wide ASPECS to estimate the properties of CO emitters, determine how clustered together CO emitters are in space, and constrain the mass of the dark matter halos that these emitters reside within. Additionally, this research works to cross-match detected CO lines with galaxies present in the survey volume, and builds the basis for future work on constraining the CO luminosity function and the density of molecular gas, building upon the success of the previous ASPECS Pilot and Large Program surveys \cite{walter2016alma, decarli2019alma}. 

\section{From the Beginning}

The initial seeds that created the large scale structure of the universe all come from tiny perturbations, observable today in the cosmic microwave background. These differences caused dark matter, the unseen matter that makes up 85\% of all matter in the universe, to clump together. These clumps of dark matter pulled in gas, creating giant clouds. Some of the gas would further collapse to create stars, and together form galaxies.

Because of this process, galaxies are composed of gas, dust, and stars in varying quantities, and so evolve in different ways. Starburst galaxies, for example, are undergoing massive amounts of star formation, and therefore are very bright and are using up the gas available to them fairly quickly. On the other hand, other galaxies have much lower masses and much lower star formation rates, that result in them depleting their gas at a much slower rate. [AWKWARD MAKE CLEARER]

These differences in behavior result in varying amounts and kinds of light being emitted from distinct types of galaxies. Young stars, for example, tend to emit most of their light in the ultraviolet, and so galaxies with large amounts of ultraviolet light tend to have had large amounts of recent star formation. Gas, on the other hand, tends to absorb the ultraviolet and visible light emitted by stars and re-emits the light at longer wavelengths, such as the infrared. Because of this, by looking at all the light that a galaxy emits, the spectral energy distribution (SED), the properties and makeup of a galaxy can be estimated.

\section{Observing Galaxies}

Parts of a galaxy are easy to observe, and other parts are not. Stars are relatively simple to observe, simply point a telescope like Hubble long enough at a point in the sky and they start to show up. Gas on the other hand, especially the cold gas that collapses to form stars, is much more difficult to detect. [CHANGE] 

Most of the gas in the universe is made up of hydrogen. Unfortunately for astronomers, molecular hydrogen, $H_2$, is only bright enough to observe when it is relatively hot, around 500 Kelvin \cite{decarli2019alma}. As the gas does not generally get this hot in the cold gas that births stars [REWORD], carbon monoxide (CO) is used instead. CO is the second most common molecule after $H_2$, but has the advantages of having bright rotational transitions that can be observed all the way to very high redshifts \cite{walter2016alma, decarli2019alma}. Additionally, the amount of $H_2$ gas can be inferred from the amount of CO that can be detected by using a straightforward conversion factor called $\alpha_{CO}$ [CITE]. 

These CO transition lines show up in the radio and (sub)mm spectrum, requiring telescopes such as ALMA and the Jansky Very Large Array (VLA) to detect them on Earth \cite{decarli2019alma}. 

\section{From Observations to Models}

All observations of a galaxy, in gamma and X-rays, visible light, infrared, radio and other wavelengths, gives us part of the SED of that galaxy. The more wavelengths a galaxy is measured in, the better constrained the SED, and the more complete our understanding of its energy distribution. The observed fluxes and values are used as inputs into programs that use simulated models to estimate a galaxy's physical parameters and total SED. These programs, such as MAGPHYS \cite{da2008simple, da2015alma}, match the observed SED to precomputed models and outputs likelihood estimates for the properties of a given galaxy, such as the star formation rate, mass, dust temperature, and more.

\section{Galaxies in Groups}

The next stage after observing and modelling individual galaxies is to look at galaxies in clusters. By looking at the distribution of galaxies in a given region of space, we can determine how densely packed together they are, which gives an insight into the amount of dark matter that surrounds a cluster of galaxies. [INSERT WHY IMPORTANT]

Galaxies are not distributed entirely randomly in the universe. On the largest scales, the universe is isotropic and homogeneous, but on smaller scales, galaxies tend to form large filaments and clusters in space. Computing the two-point correlation function and clustering parameters attempts to give a sense to how clustered a set of galaxies are. The more clustered the galaxies are, generally the more massive the halo of dark matter that surrounds them. [DOUBLE CHECK THIS]

[MAYBE INCLUDE THE BACKGROUND PHYSICS/MATH HERE FOR DM HALOS?]

This report is organized as follows. Chapter 2 focuses on the setup of the survey, the methods used to detect and cross-match lines, and the properties of those matched galaxies. Chapter 3 focuses on the clustering of CO emitters in the survey volume, and compares the clustering of sources in this survey to those of other populations of galaxies. The final chapter, chapter 4, contains the conclusions and possible future work. 

In this report, the $\Lambda$CDM cosmology is assumed, with $H_0$=70 km/s/Mpc, $\Omega_m$ = 0.3, and $\Omega_{\lambda}$ = 0.7, consistent with the results from the Planck Collaboration \cite{ade2016planck}. The analysis code used in this report is available \href{https://github.com/jacobbieker/Wide\_ASPECS}{here: https://github.com/jacobbieker/Wide\_ASPECS}. 

\chapter{Spectral Line Survey}

\section{Previous Work}

There have been two main types of surveys for searching for CO emitters in the universe, targeted and blind surveys. In targeted surveys, targets are chosen based on some type of preselection, such as star formation rate, mass, or some other parameter \cite{walter2016alma, decarli2019alma}, and are looked at specifically to see what CO emissions can be found in that galaxy or galaxies. While these surveys have given useful information on the role gas plays in these galaxies, the preselection process means that there could be a bias in our knowledge of CO. For example, these surveys could miss out on dark, gas-rich galaxies that are hard to detect, giving us an incomplete picture of the variety of galaxies with gas and CO emission in the universe. 

Blind surveys are the opposite of targeted surveys. Instead of choosing the targets and looking at them specifically, blind surveys choose an area of the sky and search for any CO emissions within that cosmic volume, allowing for any emissions above the sensitivity limit of the survey to be found. This allows for a sort of census of all the gas within a cosmic volume \cite{decarli2019alma}. 

Previous blind surveys include the COLDz survey in the GOODS-North region, the PdBI survey, as well as the first two surveys in the ASPECS program, ASPECS Pilot and ASPECS Large Program. 

The PdBI survey was a molecular line survey in the Hubble Deep Field North (HDF-N) in 3mm using the IRAM Plateau de Bure Interferometer (PdBI) \cite{decarli2014molecular}. This survey was designed to cover all CO transitions detectable above z $>$ 2, and well as covering the redshift ranges of z $<$ 0.45, and 1 $<$ z $<$ 1.9. The survey identified 17 line candidates, with most of the candidates falling in intermediate redshifts (z $<$ 3), and with 7 of the line candidates matching to galaxies in their catalog \cite{decarli2014molecular}. 

COLDz, or the CO Luminosity Density at High-z (COLDz) targeted CO(1-0) emissions from galaxies at redshifts between 1.95-2.85 and CO(2-1) emission between z = 4.91 - 6.70 using the Very Large Array (VLA)\cite{pavesi2018co}. This survey aimed to perform a similar role as ASPECS, covering a wide area of around 51 arcminute$^2$, as well as a roughly 9 arcminute$^2$ area much more deeply. This survey blindly selected starburst and massive main sequence galaxies in the GOODS-North and COSMOS fields. COLDz found seven secure CO line detections, and 57 above their S/N threshold, including many that did not have optical or near infrared counterparts, suggesting that those candidates could potentially be from a population of gas-rich, but optically dark galaxies \cite{pavesi2018co}.

The ASPECS Pilot and Large Programs both looked at regions in the Hubble Ultra Deep Field. The Pilot program looked incredibly deeply in the 1.2mm and 3mm bands at a small, one arcsecond region of the Hubble eXtremely Deep Field (XDF). The spectral coverage of the Pilot meant that it could detect CO transitions almost continuously from z = 0 to z = 8, as well as CII emissions from z = 6 to z = 8 \cite{walter2016alma}. This allowed for the characterization of CO emitters all the way to a z = 8, as well as giving constraints on the knee of the CO luminosity function \cite{walter2016alma}. 

The Large Program looked wider, but shallower, covering a roughly five square arcminute area of the Hubble Ultra Deep Field (HUDF) surrounding the ASPECS Pilot region. It was a 150 hour molecular deep field taken at 1.2mm and 3mm \cite{decarli2019alma}. The greater volume probed by the Large Program allowed for constraining the density of gas in the universe, as well as better constraints on the CO luminosity function. The survey blindly revealed the molecular gas content in normal, star-forming main-sequence galaxies \cite{decarli2019alma}. 

\section{This Survey}

The Wide ASPECS survey builds upon the Pilot and Large Program, going much wider, but also shallower. This means that Wide ASPECS cannot capture as dim CO emitters, but because of the much larger cosmic volume looked at, it can find rarer, but very bright CO emitters that did not happen to fall within the Large Program and Pilot survey bounds.

The data for this survey was observed in ALMA Cycle 5 and was observed in Band 3 (3mm) for ~20 hours in 130 pointings and six frequency settings. The survey covered ~52.5 arcminutes$^2$ of the sky at 20\% sensitivity, and ~43 arcminutes$^2$ at 50\% sensitivity. This means Wide ASPECS covers roughly five times the area of the ASPECS Large Program. This raw data was processed with the CASA ALMA pipeline into six datacubes, one for each frequency setting.

\begin{figure}[tbp]
\centering \includegraphics[width=120mm]{Wide_ASPECS_Coverage.png}
\caption{Spatial coverage of ASPECS Pilot (Green), Large Program (White), and Wide (Yellow) showing each of the individual pointings. The area covered by the Pilot is $~$1 arcmin$^2$, by the Large Program $~$5 arcmin$^2$, and by Wide ASPECS, $~$52.5 armin$^2$. The cyan box is the 50\% sensitivity extent of the survey. This region covers the area of the deepest CANDELS observations, where a wealth of ancillary data has been collected in over 30 wavelengths [CITE SURVEY DESCRIPTION].}
\label{fig:ASPECS_Coverage}
\end{figure}

\begin{figure}[tbp]
\centering
\includegraphics[width=120mm]{Wide_ASPECS_Freq.png}
\caption{Spectral and redshift coverage of Wide ASPECS in relation to the redshift distribution of galaxies within the survey field. Wide ASPECS is designed to detect CO line transitions at z$<$0.4, 1.1 $<$ z $<$ 1.8, and 2.2 $<$ z $<$ 4.4. [CITE SURVEY DESCRIPTION]}
\label{fig:ASPECS_Freq}
\end{figure}

\begin{table}[]
\caption{Redshift limits and cosmic volume probed for each CO transition observable by Wide ASPECS.}
\begin{tabular}{lllll}
Transition & $z_{low}$ & $z_{high}$ & Freq. (GHz) & Volume (Mpc$^3$) \\
\hline
1-0        & 0.0030    & 0.3694     & 115.271     & 4461             \\
2-1        & 1.0059    & 1.7387     & 230.538     & 107411           \\
3-2        & 2.0088    & 3.1080     & 345.796     & 191232           \\
4-3        & 3.0115    & 3.3771     & 461.041     & 237143          
\end{tabular}
\end{table}

\subsection{Ancillary Data}

In addition to each of those datacubes, there is a lot of ancillary data available as the GOODS-South region is one of the most studied regions of the sky. This data has been collected through combining multiple other catalogs, first described in \cite{walter2016alma} and expanded with new data in \cite{decarli2019alma}, comprising over 30 wavelength bands for over 63000 galaxies in and around the footprint of Wide ASPECS. 26251 galaxies lie within the footprint of Wide ASPECS, of which 2283 have spectroscopic redshifts, and 23968 have photometric redshifts. 

The majority of the ancillary data comes from the Hubble Space Telescope (HST) Cosmic Assembly Near-infrared Deep Extragalactic Legacy Survey (CANDELS)\cite{grogin2011candels, Koekemoer_2011}. Most of the photometric data comes from \cite{skelton20143d}, which additionally includes ground-based optical and NIR photometry from \cite{nonino2009deep, hildebrandt2006gabods, erben2005gabods, retzlaff2010great, Hsieh_2012, 2008ApJ...682..985W, cardamone2010multiwavelength}, as well as Spitzer IRAC 3.6$\mu$m, 4.5$\mu$m, 5.8$\mu$m, and 8.0$\mu$m photometry from \cite{dickinson2003evolution, elbaz2011goods, 2013ApJ...769...80A}. There is also data from the Spitzer MIPS 24$\mu$m photometric information from \cite{Whitaker_2014}, and ALMA 1.1mm data from \cite{franco2018goods}. Additional far-infrared data from Herschel PACS at 100$\mu$m and 160$\mu$m as obtained from \cite{elbaz2011goods}. Spectroscopic redshifts came mostly from the MUSE Hubble Ultra Deep Survey \cite{bacon2017, inami2017}, while more spectroscopic information was included from \cite{le2005vimos, coe2006galaxies, skelton20143d, morris2015wfc3}. Hubble grism spectroscopy was taken from the 3D-HST survey \cite{momcheva20163d}.  All these catalogs were merged into a single catalog through matching sources within 0.5 arcseconds for the photometry, and 1.0 arcseconds for the spectroscopic data. Finally, these matches were then cross-matched with morphological parameters from \cite{van2012structural}.

%26251 in footprint of Wide ASPECS
%2283 wth Spectroscopic Z
%23968 with Photometric Z

%After > 1.99 SFR cut:

%22073
%2260
%19813

\subsection{SED Fitting}

To estimate the properties of the galaxies in the catalog, the SED fitting program MAGPHYS was used \cite{da2008simple, da2015alma}. MAGPHYS computes a marginalized likelihood distribution for each of the different physical parameters of the observed galaxy through comparing the observed SED with the precomputed models. It also outputs the best-fit total SED for a given galaxy. In this report, the MAGPHYS high-z extension was used \cite{da2015alma}.

An example output from MAGPHYS is shown in Fig. \ref{fig:MAGPHYS_Example}, showing the model, the likelihood values for various physical properties of the galaxy, and the data points used to compute the SED in red. The distribution of the stellar mass, and star formation rates of all the galaxies fitted with MAGPHYS are shown in Fig. \ref{fig:MAGPHYS_Properties}. Galaxies whose computed star formation rate was at or below a $log_{10}(SFR) = -1.99$ were removed from the rest of the analysis, as this seemed to indicate a very poor MAGPHYS fit. That left a total of around 55000 galaxies in the sample, of which 22073 were in the Wide ASPECS footprint.

\begin{figure}[tbp]
\centering \includegraphics[width=120mm]{19265.pdf}
\caption{Example MAGPHYS output, from the most massive matched galaxy, ID.9 in Table \ref{table:Catalog}. The blue line is the spectrum of the galaxy without attenuation. The black line is the model used by MAGPHYS. The red dots are the data points. The bottom row of graphs show the probability distribution for various physical properties.}
\label{fig:MAGPHYS_Example}
\end{figure}

\begin{figure}[!tbp]
\centering \includegraphics[width=87mm]{Survey/MAGPHYS_SFR.png}
\caption{Distribution of star formation rate for all galaxies in the catalog. The cutoff at -1.99 is from the quality cut to remove galaxies that had very poor MAGPHYS fits.}
\label{fig:MAGPHYS_Properties}
\end{figure}

\begin{figure}[!tbp]
\centering \includegraphics[width=87mm]{Survey/MAGPHYS_Mstar.png}
\caption{Distribution of mass for all galaxies in the catalog. These galaxies have undergone the same quality cut as for the star formation rate plot.}
\label{fig:MAGPHYS_Properties}
\end{figure}

\section{Method}

As an overview, the method for finding lines, computing the fidelity, cross-matching, and determining matches is the same process as used in the other ASPECS surveys \cite{walter2016alma, decarli2019alma, gonzalez2019alma}.

\subsection{Line Search and Fidelity}

To find possible CO emitters, a search for CO lines was done with the FindClump algorithm first described in \cite{walter2016alma}. This code searches along the spectral axis with different kernel widths in order to maximize the sensitivity to signal associated with line candidates of different intrinsic widths. The widths range from 3 spectral channels up to 19 channels, with each channel being 7.813 MHz wide. The data cubes are searched for lines at any spatial position and spectral coordinate, without any prior based on data from other wavelengths. This is done to minimize any bias in the selection. FindClump returns a list of potential line candidates that then have duplicates removed, which is done by grouping line candidates based on their spatial and spectral position in the cube from each group, only storing the candidate with the highest S/N. This results in 11941 candidates at S/N$>$5.0, 1096 at S/N$>$5.5, 78 at S/N$>$6.0, and 6 at S/N$>$6.5. 

To get a sense of how many of these line candidates are false positives, the fidelity of the lines are then computed. To obtain the fidelity, a line search on the negative data cubes is performed. The negative cubes are obtained by multiplying all the values in each data cube by -1, and rerunning the line search. This catalog of negative lines is then used to compute the fidelity. The fidelity of a line at a given S/N is defined as 

$$ fideltiy(S/N) = 1 - \frac{N_{neg}(S/N)}{N_{pos}(S/N)} $$ where $N_{pos}(S/N)$ and $N_{neg}(S/N)$ are the number of positive and negative line candidates detected at that S/N and for a given line width\cite{gonzalez2019alma}.

The fidelity of the different line widths are shown in Fig. \ref{fig:Fid_map}. As can be seen, the fidelity of the lines is generally higher for wider line widths as more independent resolution elements are included \cite{decarli2019alma}. For this report, there were 16 lines at a fidelity $>$0.8, 20 at $>$0.7, 35 at $>$ 0.6, 52 at $>$ 0.5, and 69 at $>$ 0.4. The fidelity cut used for the rest of this analysis is set at 0.6, where every given line has a 60\% chance of being a real line, and leaving us with 35 line candidates. A breakdown of the number of line candidates as a function of the channel width is shown in Table \ref{table:Fid_NumTable}. 

\begin{figure}[!htbp]
\centering \includegraphics[width=120mm]{Fidelity_map.png}
\caption{Fidelity vs S/N for the different widths used by FindClump. The red dotted line shows the fidelity = 0.6 cutoff used for the analysis in this report. There are a total of 35 line candidates above that cutoff.}
\label{fig:Fid_map}
\end{figure}

\begin{table}
\centering
\caption{Number of sources, and S/N cutoffs, per channel width for the adopted fidelity cut of $>$ 0.6.}\label{table:Fid_NumTable}
\begin{tabular}{ccc}
Channels & S/N & N. Sources \\
\hline
3 & 6.28 & 3 \\
5 & 6.22 & 4 \\
7 & 6.09 & 8 \\
9 & 6.14 & 2 \\
11 & 6.12 & 6 \\
13 & 6.17 & 4 \\
15 & 6.10 & 3 \\
17 & 6.19 & 3 \\
19 & 6.00 & 2 \\
\end{tabular}
\end{table}

\subsection{Cross-matching and Redshift Determination}

Once the lines and been found, and the fidelity computed, the next step is to cross-match the CO lines to already known galaxies within the Wide ASPECS footprint. To do this, the spatial location of the line candidates were matched to galaxies that were within one arcsecond of the line candidate's location. Then, assuming that the CO line is from that matched galaxy, the CO transition was calculated. The match was only kept if the offset betwen the CO line's redshift and the galaxy's redshift, $\delta z$, was ($|\delta z| < 0.01$) for galaxies with spectroscopic redshifts, or ($|\delta z| < 0.3$) for ones with photometric redshifts. The differences in $|\delta z|$ thresholds is because spectroscopic redshifts are much more reliable and precise than photometric redshifts. 

If a line matches to more than a single galaxy, then the following steps are performed to differentiate which galaxy the line should be matched to. The first step is to calculate the CO transition for the line assuming the line is matched to each of the galaxies. Once that is determined, the difference in redshift between the CO redshift and the galaxy's redshift is calculated. The matched galaxy that gives the smallest $|\delta z|$, and whose redshift falls within the limits of Wide ASPECS, is then saved as the matched galaxy for that line candidate. 

If there is not a match to a galaxy in the catalog, or if the galaxy's redshift is incompatible with the CO line identification, then the line identification is performed by a bootstrap method, where the probability of a line candidate being one of CO(1-0), CO(2-1), CO(3-2), or CO(4-3) is proportional to the volume of the universe sampled in each of those transitions \cite{walter2016alma, decarli2019alma}.

\section{Results}

The mass versus star formation rate is shown for the general galactic population and the ASPECS sources in Fig. \ref{fig:Cross_match}. For comparison, the Schreiber et al. 2015 \cite{schreiber2015herschel} and Whitaker et al. 2014 \cite{Whitaker_2014} main sequence fits are plotted as well. As can be seen, two of the matched galaxies are above the main sequence and are massive, star-forming galaxies that are the expected type of galaxies for this survey to match to. The other galaxy is much less massive than expected. 

\begin{figure}[!htbp]
\centering \includegraphics[width=120mm]{Survey/No_Cut_Mstar_vs_SFR_all_closest_sep_1_0_sn_fid_60.png}
\caption{Mass vs Star Formation Rate for the matched galaxies. Red points are matched galaxies with spectroscopic redshifts, while blue points are matched galaxies with photometric redshifts. Grey points are all the galaxies in the catalog. The green lines are the galaxy main sequence fits from \cite{schreiber2015herschel}. The yellow line is from \cite{Whitaker_2014}'s galactic main sequence fit, where the solid line means it is computed within the range mentioned in the paper, while dotted means that the values are extrapolated from the paper to higher, or lower, redshifts. Two of the three matches are on the upper edge of the main sequence, while the third match has a much lower mass and star formation rate than expected for this survey. Error bars are the 16/84th percentile outputs from MAGPHYS.}
\label{fig:Cross_match}
\end{figure}

\subsection{Catalog}

The final catalog of CO line emitters is shown in Table \ref{table:Catalog}. 3 lines match to galaxies in the catalog. Of those three matches, only ID.1's match has a spectroscopic redshift. The redshift distribution of the whole catalog is shown in Fig. \ref{fig:cat_red}, and some of the physical properties of the three matched lines is shown in Table. \ref{table:matched_gal}. 

\begin{table}
\centering
\caption{RA and DEC are in the J2000 system. $v_{CO}$ is the observed frequency of the source in GHz. CO Tran. is the estimated CO transition. If available, $\delta z$ is the redshift difference between the source and the matched galaxy. S/N is the signal-to-noise of the source. Match tells whether there is a match to a galaxy in the catalog. Only ID.1 has a matched galaxy with a spectroscopic redshift.}
\begin{tabular}{ccccccccc}
ID & RA & DEC & $v_{CO}$ & CO Tran. & $z_{CO}$ & $\delta z$ & S/N & Match? \\
\hline
ID.1 & 53.14886 & -27.82118 & 96.701 & 3-2 & 2.576 & 0.006 & 7.31 & Y \\
ID.2 & 53.19145 & -27.76985 & 91.657 & 2-1 & 1.515 & -- & 6.82 & N \\
ID.3 & 53.14138 & -27.84409 & 96.834 & 4-3 & 3.761 & -- & 6.72 & N \\
ID.4 & 53.19242 & -27.78342 & 106.72 & 3-2 & 2.24 & -- & 6.63 & N \\
ID.5 & 53.16066 & -27.76629 & 86.677 & 2-1 & 1.66 & -0.237 & 6.6 & Y \\
ID.6 & 53.16003 & -27.76258 & 103.36 & 3-2 & 2.346 & -- & 6.6 & N \\
ID.7 & 53.13447 & -27.74976 & 102.719 & 3-2 & 2.366 & -- & 6.49 & N \\
ID.8 & 53.11066 & -27.82727 & 85.654 & 3-2 & 3.037 & -- & 6.45 & N \\
ID.9 & 53.11881 & -27.78291 & 104.501 & 3-2 & 2.309 & 0.081 & 6.43 & Y \\
ID.10 & 53.13921 & -27.75352 & 86.67 & 3-2 & 2.99 & -- & 6.43 & N \\
ID.11 & 53.07696 & -27.8251 & 90.891 & 2-1 & 1.536 & -- & 6.42 & N \\
ID.12 & 53.12766 & -27.76666 & 99.553 & 4-3 & 3.631 & -- & 6.42 & N \\
ID.13 & 53.07796 & -27.80182 & 98.725 & 4-3 & 3.67 & -- & 6.36 & N \\
ID.14 & 53.13923 & -27.78228 & 94.407 & 2-1 & 1.442 & -- & 6.35 & N \\
ID.15 & 53.04266 & -27.79322 & 103.173 & 3-2 & 2.352 & -- & 6.3 & N \\
ID.16 & 53.116 & -27.84624 & 97.569 & 4-3 & 3.725 & -- & 6.29 & N \\
ID.17 & 53.09146 & -27.8497 & 106.235 & 3-2 & 2.255 & -- & 6.27 & N \\
ID.18 & 53.18748 & -27.81369 & 89.704 & 1-0 & 0.285 & -- & 6.27 & N \\
ID.19 & 53.11238 & -27.75628 & 88.735 & 3-2 & 2.897 & -- & 6.26 & N \\
ID.20 & 53.12915 & -27.79241 & 91.54 & 2-1 & 1.518 & -- & 6.22 & N \\
ID.21 & 53.1179 & -27.8184 & 92.876 & 2-1 & 1.482 & -- & 6.21 & N \\
ID.22 & 53.06549 & -27.84266 & 104.094 & 3-2 & 2.322 & -- & 6.21 & N \\
ID.23 & 53.1612 & -27.76049 & 94.618 & 2-1 & 1.437 & -- & 6.19 & N \\
ID.24 & 53.11182 & -27.82032 & 99.522 & 4-3 & 3.633 & -- & 6.19 & N \\
ID.25 & 53.07121 & -27.82724 & 88.954 & 3-2 & 2.887 & -- & 6.17 & N \\
ID.26 & 53.19977 & -27.8282 & 100.055 & 3-2 & 2.456 & -- & 6.16 & N \\
ID.27 & 53.06316 & -27.82356 & 84.623 & 3-2 & 3.086 & -- & 6.15 & N \\
ID.28 & 53.09788 & -27.76003 & 90.618 & 3-2 & 2.816 & -- & 6.13 & N \\
ID.29 & 53.16282 & -27.84445 & 90.571 & 3-2 & 2.818 & -- & 6.12 & N \\
ID.30 & 53.09665 & -27.81229 & 95.259 & 2-1 & 1.42 & -- & 6.12 & N \\
ID.31 & 53.13844 & -27.80491 & 102.376 & 3-2 & 2.378 & -- & 6.12 & N \\
ID.32 & 53.16516 & -27.82257 & 104.931 & 3-2 & 2.295 & -- & 6.11 & N \\
ID.33 & 53.19483 & -27.81481 & 98.873 & 4-3 & 3.663 & -- & 6.1 & N \\
ID.34 & 53.08948 & -27.78102 & 86.068 & 3-2 & 3.018 & -- & 6.1 & N \\
ID.35 & 53.14828 & -27.84444 & 87.302 & 3-2 & 2.961 & -- & 6.1 & N \\
\end{tabular}
\end{table}\label{table:Catalog}

\begin{figure}[tbp]
\centering \includegraphics[width=120mm]{Survey/redshift_catalog.png}
\caption{Redshift distribution for the CO redshifts.}
\label{fig:cat_red}
\end{figure}

\begin{table}
\caption{This shows some of the physical parameters for the 3 matched line candidates. Sep is the separation in arcseconds between the CO line and the galaxy. $z_{catalog}$ is the master catalog's redshift for the galaxy.}
\begin{tabular}{ccccccccccccccc}
ID & Trans. & $z_{CO}$ & $z_{catalog}$ & $\delta z$ & Spec & S/N & Sep & Log(SFR) & Log(M*) \\
\hline
ID.1 & 3-2 & 2.576 & 2.582 & 0.006 & Y & 7.31 & 0.2824 & $2.782_{-0.005}^{+0.005}$ & $11.31_{-0.0}^{+0.0}$  \\
ID.5 & 2-1 & 1.66 & 1.423 & -0.237 & N & 6.6 & 0.9723 & $-0.453_{-0.255}^{+0.255}$ & $8.017_{-0.175}^{+0.175}$ \\
ID.9 & 3-2 & 2.309 & 2.39 & 0.081 & N & 6.43 & 0.4861 & $1.307_{-0.0}^{+0.0}$ & $9.757_{-0.0}^{+0.0}$ \\
\end{tabular}\label{table:matched_gal}
\end{table}

In addition, the line candidates are shown overlaid over other wavelengths in the Appendix \ref{sec:A1}.

\section{Discussion}

There are significantly less matches than was expected. Only $~$9\% percent of the lines match to galaxies. Of those that do, only two matched to the kind of galaxies that were expected, which are the most massive, most star forming galaxies in the field. 

Possible explanations include issues with MAGPHYS' fitting of the galaxies in the catalog, as many of the galaxies seem poorly constrained. 

\chapter{Clustering of CO emitters}

Galaxies are not distributed randomly in the universe. On the largest scales, the universe is isotropic and homogeneous, but on smaller scales, galaxies tend to cluster in space. One way to quantify how densely packed a population of galaxies is to compute the so-called two-point correlation function. This allows for the measuring of the mass of the dark matter halos in which these galaxies reside \cite{hickox2011clustering}. For a certain redshift, the more clustered a population of galaxies is, the more massive the halo that they are in.

The clustering of different populations of galaxies has been constrained by previous works, which some are summarized in Fig. \ref{fig:Hickox_compare}, where the correlation length parameter, $r_0$, which is a proxy for the amplitude of the clustering, is shown as a function of redshift for different populations. For example, \cite{hickox2011clustering} looked at the clustering of unobscured quasars and obscured quasars. \cite{10.1111/j.1365-2966.2011.20303.x} looks at the clustering in SMGs and compares their clustering to clustering found for other sets of astronomical objects, as can be seen in Fig. \ref{fig:Hickox_compare}. \cite{hickox2011clustering} found, in the Bo\"otes field, an $r_0$ for unobscured quasars to be $5.6 \pm 0.8$ at 0$<$z$<$1, and for obscured quasars $6.0 \pm 1.0 $. In \cite{10.1111/j.1365-2966.2011.20303.x}, submillimeter galaxies (SMGs) in the Extended Chandra Deep Field South were found to have a $r_0$ of $7.7_{-2.3}^{+1.8}$ at a 1$<$z$<$3. Other sources, such as \cite{adelberger2005spatial}, found for Lyman Break Galaxies (LBGs) $r_0$ values of $4.5 \pm 0.6$ at $z = 1.7$, $4.2 \pm 0.5$ at $z = 2.2$, and $4.0 \pm 0.6$ at $z = 2.9$, while \cite{ross2009clustering} found, for quasars in the Sload Digital Sky Survey, an $r_0 = 5.45_{-0.45}^{+0.35}$ for z$\leq$2.2, both shown in Fig. \ref{fig:Hickox_compare}. 

\begin{figure}[!htbp]
\centering \includegraphics[width=87mm]{clustering/Hickox2012_Compare.png}
\caption{$r_0$ values for a variety of celestial objects as a function of redshift. The grey dotted lines show the evolution of $r_0$ for dark matter haloes of different masses. Figure adapted from \cite{10.1111/j.1365-2966.2011.20303.x}.}
\label{fig:Hickox_compare}
\end{figure}

\section{Method}

%The clustering is computed through a two point correlation function. %The first step s to calculate the angular correlation function $ \omega(\theta)$. Once $ \omega(\theta)$ is computed, the second step is to obtain the two-point correlation function parameters of $r_0$ and $\gamma$.

The clustering is computed through a two point correlation function. The two-point correlation function, $ \xi(r)$ is the the probability of finding a galaxy at a separation $r$ from another randomly chosen galaxy in a volume $dV$ above Poisson, such that $$ dP = n(1 + \xi(r))dV $$, where $n$ is the mean space density of the galaxies in the universe\cite{hickox2011clustering}, and $\xi(r)$ is normally modeled as a power-law such that $\xi(r)$ = $(\frac{r}{r_0})^{-\gamma}$. In practice, since the 3D separation $r$ between objects is hard to measure, the projected separations ($R$) or angular separations ($\theta$) are used instead. Here, we focus in the computation of the angular auto-correlation function $\omega(\theta)$. This quantity can be computed using the estimator from \cite{1993ApJ...412...64L}, where $$ \omega(\theta) = \frac{1}{RR}(DD-2DR + RR)$$, where $DD$, $DR$, and $RR$ are the number of data-data, data-random, and random-random galaxy pairs at a separation $\theta$, where each of the three collections is normalized to sum to 1 \cite{hickox2011clustering}. Here the 'data' catalog is represented by our actual CO emitter catalog of 35 sources, while the 'random' catalog is created by randomly distributing 20000 sources over the geometry of the survey, mimicking exactly the same area where CO emitters were detected. The distribution of data and random sources are shown in Fig. \ref{fig:Clustering_points}.

%I think you are mixing a bit the things here. All the things about the fitting and getting r0 and gamma, z distribution, should be explained after to show all your clustering measurements.

%Then here
%1. first put that you checked your code using 2 random catalogs as you wrote below. And show the figure.

In order to check our code for the clustering computation, we created 2 additional random catalogs, of 20000 points each, and their angular correlation computed to show that it is consistent with zero, shown in Fig.\ref{fig:random_points}. The error bars for $\omega(\theta)$ are computed using possion errors for small statistics \cite{1986ApJ...303..336G}.

\begin{figure}[!htbp]
\centering \includegraphics[width=100mm]{clustering/5Random_vs_Random_10000_bin9.png}
\caption{Angular correlation two sets of random points used in this analysis, showing that the angular correlation is consistent with zero, as would be expected for cross-correlating two sets of uniformly distributed points. The red line shows $\omega(\theta) = 0$.}
\label{fig:random_points}
\end{figure}


%2. put that you use your CO catalog (with fidelity >0.6) and the random catalog, and show your figure 3.5. Mention that you cut in z (put the redshift range used), Mention that you measured this in logaritmically spaced bins and put those numbers, and that you defined your lower bin as the minimum distance between objects in the catalog (although in your plot it looks like you are not doing that).

\begin{figure}[!htbp]
\centering \includegraphics[width=120mm]{PDFS/NX_V_Y_Sources_20000.png}
\caption{Random points, in grey, and sources, in blue, from the $>$ 0.6 fidelity cut. The random points are distributed uniformly throughout the Wide ASPECS footprint.}
\label{fig:Clustering_points}
\end{figure}

%4. Then put what you wrote: Once the angular correlation function is found...angular correlations [7].
%Explain that you fit for all, and then for only positive values.

Once the angular correlation function is found, a power-law model is fitted following $$\omega(\theta) = A\theta^{-\beta} $$ where $\beta$ = 0.8, a value used for many other galaxy angular correlations \cite{hickox2011clustering}. The data is fitted twice, once for all the binned data, and once fitting only to the bins with positive values.

\subsection{Obtaining $r_0$ and $\gamma$}

%5. Then put the section 3.2.2 until the part where you describe all the equations to compute r0 and gamma. Also here explain that you used the CO z distribution, and the gaussian fit to it. Then report the r0 and gamma values that you measured. And mention somewhere that you have low statistcs.

Two equations are used to convert the $A$ and $\beta$ to real-space $r_0$ and $\gamma$, 

$$ \gamma = \beta + 1 $$ and $$ A = H_{\gamma}\frac{\int_{0}^{\inf} (dN_1/dz)(dN_2/dz)E_z\chi^{1 - \gamma} dz}{[\int_{0}^{\inf} (dN_1/dz)dz][\int_{0}^{\inf} (dN_2/dz)dz]}r_0^{\gamma}$$

where $H_{\gamma} = \Gamma(0.5)\Gamma(0.5[\gamma -1])\Gamma(0.5\gamma)$, with $\Gamma$ being the gamma function, $\chi$ the radial comoving distance, $dN_{1,2}/dz$ are the redshift distributions of the samples, where in the case of autocorrelation are equal to each other, and $E_z = H_z/c = dz/d\chi$ \cite{hickox2011clustering}. The Hubble parameter $H_z$ can be found from
$$H_z^2 = H_0^2[\Omega_m(1+z)^3 + \Omega_{\lambda}]$$ \cite{hickox2011clustering}.

For this analysis, the redshift distribution was taken from the CO redshifts of the lines, and the calculations were performed over the range z = 1.5 to 3.5, as this is the range where most of the CO candidates seem to lie. To calculate $dN_{1,2}/dz$, the redshift distribution of the CO lines was fitted with a Gaussian, and the integral was taken over the Gaussian fit. The angular correlation function was computed over the range of 8.39 to 582.10 arcseconds, using logarithmically spaced bins. Because there are only 35 line candidates in the catalog, the clustering measurements are quite noisy. The final calculated values are $r_0 = 6.96_{-0.82}^{+0.75}$, and $\gamma = 1.8$. 

\begin{figure}[!htbp]
% explain how bins can be negative and what this means for fitting only positive bins
% Bins can be negative through.... 0, so not negative I guess? 
\centering \includegraphics[width=90mm]{clustering_two/Data_vs_Random_20000_bin6_sn0_6_NFalse.png}
\caption{Angular correlation function for 6 bins for the chosen fidelity cut of 0.6. Red is the fit $\omega(\theta) = A\theta^{-0.8}$ to all the bins, while the green line is the fit to only bins with positive values. This is the final binning used for the analysis. }
\label{fig:Angular_binnings}
\end{figure}
%6. Then put a discussion section, and comment about if the r0 values that you got is higher or lower to other populations (as you can see in Fig. 3.1). Then say that you did the same procedure for other fidelity cuts to compare, and show your 4panel plot and comment about the fact that signal goes up with fidelity.

\section{Discussion}

In comparison to previous results, the $r_0$ of the CO emitters here are quite similar to the values for the SMGs found in \cite{10.1111/j.1365-2966.2011.20303.x}, and quasars in the Bo\"otes field \cite{hickox2011clustering}, while being slightly more clustered than the quasars found in \cite{ross2009clustering}. In comparison to other populations of celestial objects shown in Fig. \ref{fig:Hickox_compare}, the clustering for CO emitters seems to be stronger than for the Lyman Break Galaxies in \cite{adelberger2005spatial}, which covers the same redshift range of 1.5$<$z$<$3.5. This implies that the dark matter halos surrounding these CO emitters would be in the range very roughly between $10^{12} h^{-1}M_{\odot}$ and $10^{13} h^{-1}M_{\odot}$.

% Add more about the effects of binning, negative/0 values 

The two point correlation function was also computed for other fidelity cuts. As can be seen in Fig. \ref{fig:Angular_correlation}, as the fidelity increases, $A$ increases as well. When converted to $r_0$, the fidelity cuts result in $r_0$ values of $7.8_{-7.8}^{+5.78}$ for fidelity $>$ 0.7, $6.96_{-0.82}^{+0.74}$ for fidelity $>$ 0.6, $0.0_{-0.0}^{+4.32}$ for fidelity $>$ 0.5, and $r_0 = 0.0_{-0.0}^{+3.83}$ for fidelity $>$ 0.4. This increasing $r_0$ as the fidelity increases suggests that there are real sources in the sample.

\begin{figure}[!htbp]
% Use Symlog to plot it then, will keep negative values and such
\centering \includegraphics[width=160mm]{clustering_two/Log_4Panel_Data_Vs_Random_bin6_NFalse_Num20000.png}
\caption{Angular correlation function for various fidelity cuts. The bins increase logarithmically from 8.39 arcsecs to 582.10 arcseconds. The red lines are from fitting $\omega(\theta) = A\theta^{-0.8} $ to all of the bins. The green dashed line is from fitting that same equation only to bins that had a positive value. The yellow dashed line is the fit from the fidelity $>$ 0.4 catalog. As the fidelity goes up, the $A$ value increases as well, indicating stronger clustering. The few line candidates means that the results are quite noisy, and are sensitive to the bins chosen.}
\label{fig:Angular_correlation}
\end{figure}

%\begin{figure}[!htbp]
% Use Symlog to plot it then, will keep negative values and such
%\centering \includegraphics[width=160mm]{clustering_two/4Panel_Data_Vs_Random_bin6_NFalse_Num20000.png}
%\caption{Angular correlation function for various fidelity cuts. The bins increase logarithmically from 8.39 arcsecs to 582.10 arcseconds. The red lines are from fitting $\omega(\theta) = A\theta^{-0.8} $ to all of the bins. The green dashed line is from fitting that same equation only to bins that had a positive value. The yellow dashed line is the fit from the fidelity $>$ 0.4 catalog. As the fidelity goes up, the $A$ value increases as well, indicating stronger clustering. The few line candidates means that the results are quite noisy, and are sensitive to the bins chosen.}
%\label{fig:Angular_correlation_linear}
%\end{figure}

Besides the different fidelity cuts, different distance binnings also are included to study their effects on the final $r_0$ results. The results seem to be very dependent on the binning. While the values for the only positively fitted bins does not change dramatically as the binning changes, the number of bins does make a large difference in the final $A$, and therefore $r_0$ values, as can be seen in Figures \ref{fig:Angular_bin_8} and \ref{fig:Angular_bin_5} in comparison to Fig. \ref{fig:Angular_binnings}. The range in the $r_0$ values underlines how the low statistics available makes the clustering measurement very noisy, although all the $r_0$ values for the fidelity $>$ 0.6 sample are consistent with each other. 

\begin{figure}[!htbp]
\centering \includegraphics[width=90mm]{clustering_two/Data_vs_Random_20000_bin8_sn0_6_NFalse.png}
\caption{Angular correlation function for 8 bins  for fidelity $>$ 0.6. In this case, an increase in the number of bins decreases the $r_0$ value, although it is still consistent with the original $r_0$ value from \ref{fig:Angular_binnings}.}
\label{fig:Angular_bin_8}
\end{figure}

\begin{figure}[!htbp]
\centering \includegraphics[width=90mm]{clustering_two/Data_vs_Random_20000_bin10_sn0_6_NFalse.png}
\caption{Angular correlation function for 10 bins for fidelity $>$ 0.6. In this case, further increasing the number of bins also decreases the $r_0$ value, which is still consistent with the $r_0$ found in \ref{fig:Angular_binnings}. This could be a result of the very low statistics and so extremely noisy measurements.}
\label{fig:Angular_bin_5}
\end{figure}

The existence of negative values in the correlation function means that there are fewer numbers of data-data pairs in that bin compared with expectation, which could happen when a field is underdense. However, in that case, we would expect that all the points would be systematically negative, so in this case it is most likely related with the extremely high noise in the measurement. This could explain why, as the number of bins is increased, $r_0$ continues to decrease. 

These results do show that there is a larger $r_0$ as the fidelity cut becomes higher, suggesting that there are real sources in the sample. On the other hand, the clustering measurement is still very noisy and future work will be required to get a better final clustering measurement.

% Random points: 582.10 arcseconds, DD points, 516 arcseconds



\chapter{Conclusion}

[INCLUDE CONCLUSION]

\section{Future Work}

[ADD HOW TO CONVERT TO LUMINOSITIES? AND GAS DENSITY? OR REFERENCE THE PAPERS AT LEAST]
The future work includes improving the clustering measurement by performing the cross-correlation with the galaxy catalog, which should improve the S/N of the clustering measurement versus the current auto-correlation. In addition, the CO luminosity function and density of molecular gas could also be computed from the data available, through converting the line fluxes to luminosities, for the CO luminosity function, then continuing and converting the luminosities to gas densities. Additional work could be done to verify the CO lines matched to galaxies,and investigate why the lines seem to match to so few galaxies, and galaxies that are so much lower on the Mstar vs SFR plot than would be expected. 

\appendix
\chapter{Appendix}

\begin{table}
\centering
\caption{Line fidelity as a function of S/N and line width.}
\begin{tabular}{cccccccccc}
\hline
S/N & 3 chn & 5 chn & 7 chn & 9 chn & 11 chn & 13 chn & 15 chn & 17 chn & 19 chn \\
\hline
5.85 & 0.09 & 0.07 & 0.15 & 0.12 & 0.11 & 0.1 & 0.17 & 0.14 & 0.26 \\
5.95 & 0.15 & 0.14 & 0.3 & 0.23 & 0.24 & 0.2 & 0.31 & 0.23 & 0.48 \\
6.05 & 0.26 & 0.27 & 0.51 & 0.41 & 0.44 & 0.36 & 0.51 & 0.37 & 0.72 \\
6.15 & 0.4 & 0.45 & 0.71 & 0.62 & 0.67 & 0.56 & 0.7 & 0.54 & 0.87 \\
6.25 & 0.56 & 0.66 & 0.86 & 0.79 & 0.84 & 0.75 & 0.85 & 0.7 & 0.95 \\
6.35 & 0.71 & 0.82 & 0.94 & 0.9 & 0.93 & 0.87 & 0.93 & 0.82 & 0.98 \\
6.45 & 0.83 & 0.91 & 0.97 & 0.96 & 0.97 & 0.94 & 0.97 & 0.9 & 0.99 \\
6.55 & 0.91 & 0.96 & 0.99 & 0.98 & 0.99 & 0.97 & 0.99 & 0.95 & 1.0 \\
6.65 & 0.95 & 0.98 & 1.0 & 0.99 & 1.0 & 0.99 & 0.99 & 0.98 & 1.0 \\
6.75 & 0.98 & 0.99 & 1.0 & 1.0 & 1.0 & 1.0 & 1.0 & 0.99 & 1.0 \\
6.85 & 0.99 & 1.0 & 1.0 & 1.0 & 1.0 & 1.0 & 1.0 & 0.99 & 1.0 \\
6.95 & 0.99 & 1.0 & 1.0 & 1.0 & 1.0 & 1.0 & 1.0 & 1.0 & 1.0 \\
7.05 & 1.0 & 1.0 & 1.0 & 1.0 & 1.0 & 1.0 & 1.0 & 1.0 & 1.0 \\
7.15 & 1.0 & 1.0 & 1.0 & 1.0 & 1.0 & 1.0 & 1.0 & 1.0 & 1.0 \\
7.25 & 1.0 & 1.0 & 1.0 & 1.0 & 1.0 & 1.0 & 1.0 & 1.0 & 1.0 \\
7.35 & 1.0 & 1.0 & 1.0 & 1.0 & 1.0 & 1.0 & 1.0 & 1.0 & 1.0 \\
7.45 & 1.0 & 1.0 & 1.0 & 1.0 & 1.0 & 1.0 & 1.0 & 1.0 & 1.0 \\
7.55 & 1.0 & 1.0 & 1.0 & 1.0 & 1.0 & 1.0 & 1.0 & 1.0 & 1.0 \\
7.65 & 1.0 & 1.0 & 1.0 & 1.0 & 1.0 & 1.0 & 1.0 & 1.0 & 1.0 \\
7.75 & 1.0 & 1.0 & 1.0 & 1.0 & 1.0 & 1.0 & 1.0 & 1.0 & 1.0 \\
7.85 & 1.0 & 1.0 & 1.0 & 1.0 & 1.0 & 1.0 & 1.0 & 1.0 & 1.0 \\
7.95 & 1.0 & 1.0 & 1.0 & 1.0 & 1.0 & 1.0 & 1.0 & 1.0 & 1.0 \\
\end{tabular}
\end{table}\label{table:Fid_Table}

\section{Line Candidates}\label{sec:A1}

The following plots show the 35 CO line candidates in various wavelengths. The red contours show the  +2$\sigma$,4$\sigma$,6$\sigma$,and 8$\sigma$ values in the datacubes. 

\begin{figure}[tbp]
\centering \includegraphics[width=160mm]{Matched/ASPECS_Cutout_0.jpg}
\caption{ID.1. Red contours are the CO detections at +2$\sigma$,4$\sigma$,6$\sigma$,8$\sigma$, and 10$\sigma$. }
\label{fig:Match_One}
\end{figure}

\begin{figure}[tbp]
\centering \includegraphics[width=160mm]{Matched/ASPECS_Cutout_1.jpg}
\caption{ID.2}
\label{fig:Match_Two}
\end{figure}


\begin{figure}[tbp]
\centering \includegraphics[width=160mm]{Matched/ASPECS_Cutout_2.jpg}
\caption{ID.3}
\label{fig:Match_Three}
\end{figure}

\begin{figure}[tbp]
\centering \includegraphics[width=160mm]{Matched/ASPECS_Cutout_3.jpg}
\caption{ID.4}
\label{fig:Match_Three}
\end{figure}

\begin{figure}[tbp]
\centering \includegraphics[width=160mm]{Matched/ASPECS_Cutout_4.jpg}
\caption{ID.5}
\label{fig:Match_Four}
\end{figure}

\begin{figure}[tbp]
\centering \includegraphics[width=160mm]{Matched/ASPECS_Cutout_5.jpg}
\caption{ID.6}
\label{fig:Match_Five}
\end{figure}

\begin{figure}[tbp]
\centering \includegraphics[width=160mm]{Matched/ASPECS_Cutout_6.jpg}
\caption{ID.7}
\label{fig:Match_Three}
\end{figure}

\begin{figure}[tbp]
\centering \includegraphics[width=160mm]{Matched/ASPECS_Cutout_7.jpg}
\caption{ID.8}
\label{fig:Match_Three}
\end{figure}

\begin{figure}[tbp]
\centering \includegraphics[width=160mm]{Matched/ASPECS_Cutout_8.jpg}
\caption{ID.9}
\label{fig:Match_Three}
\end{figure}

\begin{figure}[tbp]
\centering \includegraphics[width=160mm]{Matched/ASPECS_Cutout_9.jpg}
\caption{ID.10}
\label{fig:Match_Three}
\end{figure}


\begin{figure}[tbp]
\centering \includegraphics[width=160mm]{Matched/ASPECS_Cutout_10.jpg}
\caption{ID.11}
\label{fig:Match_Three}
\end{figure}

\begin{figure}[tbp]
\centering \includegraphics[width=160mm]{Matched/ASPECS_Cutout_11.jpg}
\caption{ID.12}
\label{fig:Match_Three}
\end{figure}

\begin{figure}[tbp]
\centering \includegraphics[width=160mm]{Matched/ASPECS_Cutout_12.jpg}
\caption{ID.13}
\label{fig:Match_Three}
\end{figure}

\begin{figure}[tbp]
\centering \includegraphics[width=160mm]{Matched/ASPECS_Cutout_13.jpg}
\caption{ID.14}
\label{fig:Match_Three}
\end{figure}

\begin{figure}[tbp]
\centering \includegraphics[width=160mm]{Matched/ASPECS_Cutout_14.jpg}
\caption{ID.15}
\label{fig:Match_Three}
\end{figure}

\begin{figure}[tbp]
\centering \includegraphics[width=160mm]{Matched/ASPECS_Cutout_15.jpg}
\caption{ID.16}
\label{fig:Match_Three}
\end{figure}

\begin{figure}[tbp]
\centering \includegraphics[width=160mm]{Matched/ASPECS_Cutout_16.jpg}
\caption{ID.17}
\label{fig:Match_Three}
\end{figure}

\begin{figure}[tbp]
\centering \includegraphics[width=160mm]{Matched/ASPECS_Cutout_17.jpg}
\caption{ID.18}
\label{fig:Match_Three}
\end{figure}

\begin{figure}[tbp]
\centering \includegraphics[width=160mm]{Matched/ASPECS_Cutout_18.jpg}
\caption{ID.19}
\label{fig:Match_Three}
\end{figure}

\begin{figure}[tbp]
\centering \includegraphics[width=160mm]{Matched/ASPECS_Cutout_19.jpg}
\caption{ID.20}
\label{fig:Match_Three}
\end{figure}

\begin{figure}[tbp]
\centering \includegraphics[width=160mm]{Matched/ASPECS_Cutout_20.jpg}
\caption{ID.21}
\label{fig:Match_Three}
\end{figure}

\begin{figure}[tbp]
\centering \includegraphics[width=160mm]{Matched/ASPECS_Cutout_21.jpg}
\caption{ID.22}
\label{fig:Match_Three}
\end{figure}

\begin{figure}[tbp]
\centering \includegraphics[width=160mm]{Matched/ASPECS_Cutout_22.jpg}
\caption{ID.23}
\label{fig:Match_Three}
\end{figure}

\begin{figure}[tbp]
\centering \includegraphics[width=160mm]{Matched/ASPECS_Cutout_23.jpg}
\caption{ID.24}
\label{fig:Match_Three}
\end{figure}

\begin{figure}[tbp]
\centering \includegraphics[width=160mm]{Matched/ASPECS_Cutout_24.jpg}
\caption{ID.25}
\label{fig:Match_Three}
\end{figure}

\begin{figure}[tbp]
\centering \includegraphics[width=160mm]{Matched/ASPECS_Cutout_25.jpg}
\caption{ID.26}
\label{fig:Match_Three}
\end{figure}

\begin{figure}[tbp]
\centering \includegraphics[width=160mm]{Matched/ASPECS_Cutout_26.jpg}
\caption{ID.27}
\label{fig:Match_Three}
\end{figure}

\begin{figure}[tbp]
\centering \includegraphics[width=160mm]{Matched/ASPECS_Cutout_27.jpg}
\caption{ID.28}
\label{fig:Match_Three}
\end{figure}

\begin{figure}[tbp]
\centering \includegraphics[width=160mm]{Matched/ASPECS_Cutout_28.jpg}
\caption{ID.29}
\label{fig:Match_Three}
\end{figure}

\begin{figure}[tbp]
\centering \includegraphics[width=160mm]{Matched/ASPECS_Cutout_29.jpg}
\caption{ID.30}
\label{fig:Match_Three}
\end{figure}

\begin{figure}[tbp]
\centering \includegraphics[width=160mm]{Matched/ASPECS_Cutout_30.jpg}
\caption{ID.31}
\label{fig:Match_Three}
\end{figure}

\begin{figure}[tbp]
\centering \includegraphics[width=160mm]{Matched/ASPECS_Cutout_31.jpg}
\caption{ID.32}
\label{fig:Match_Three}
\end{figure}

\begin{figure}[tbp]
\centering \includegraphics[width=160mm]{Matched/ASPECS_Cutout_32.jpg}
\caption{ID.33}
\label{fig:Match_Three}
\end{figure}

\begin{figure}[tbp]
\centering \includegraphics[width=160mm]{Matched/ASPECS_Cutout_33.jpg}
\caption{ID.34}
\label{fig:Match_Three}
\end{figure}

\begin{figure}[tbp]
\centering \includegraphics[width=160mm]{Matched/ASPECS_Cutout_34.jpg}
\caption{ID.35}
\label{fig:Match_Three}
\end{figure}


For example, star formation has not stayed at a constant rate through cosmic time. Generally, galaxies had a higher star formation rate (SFR) in the past than today, with the peak star formation happening around 10 billion years ago, as seen in Fig. \ref{fig:SFR_History}. To sustain this star formation rate, there needs to be enough gas available to make new stars, and it has been observed that the fraction of gas in galaxies tends to increase with increased redshift [CITE, Scoville et al 2016]. 

\begin{figure}[tbp]
\centering \includegraphics[width=100mm]{results_text/figure9a.jpg}
\caption{Comoving star formation rate density as a function of redshift, taken from observations in the ultraviolet and infrared, showing the  star formation history as traced by stars themselves. This report focuses on learning more about the gas that would create these stars. By observing molecular gas we can constrain star formation rate from a different and complementary approach, because we are not observing stars directly but instead we are observing regions where stars form. Figure taken from \cite{madau2014cosmic}.}
\label{fig:SFR_History}
\end{figure}

\bibliographystyle{unsrt}
\bibliography{refs}

\end{document}
